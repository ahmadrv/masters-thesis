\chapter{مقدمه}
در این فصل، تحت یک مقدمه، به بیان مسئله و ضرورت وجود محک شبیه‌سازها و پس از آن اشاره‌ای به شکاف‌های پژوهشی، روش پیشنهادی برای برطرف‌کردن شکاف و در نهایت به دستاوردها و نتایج می‌پردازیم.
\section{بیان مسئله}
محک شبیه‌سازهای مدار کوانتومی به دلیل تحول فناوری رایانش کوانتومی، ضروری است. نیاز اصلی به محک ازاین‌جهت ناشی می‌شود که لازم است عملکرد دستگاه‌ها و شبیه‌سازهای مختلف رایانش کوانتومی به طور سامان‌مند ارزیابی و مقایسه شود. با ادامهٔ توسعهٔ سخت‌افزار و الگوریتم‌های کوانتومی، محک‌های کارآمدتر، یک چارچوب ثابت برای ارزیابی بهبودها، شناسایی نقاط ضعف و هدایت تحقیقات و توسعه را فراهم می‌کنند. باتوجه‌به رویکردهای متنوع برای اصلاح و کاهش خطا در رایانش کوانتومی، محک‌ها می‌توانند اثربخشی این رویکرد‌ها را تحت شرایط مختلف برجسته کنند. این امر به پژوهشگران و توسعه‌دهندگان این امکان را می‌دهد که روش‌های خود را اصلاح کرده و دقت و قابلیت اطمینان کلی محاسبات کوانتومی را بهبود بخشند\cite{lubinski_application-oriented_2023}.

برخلاف پیشرفت‌های قابل‌توجه در رایانش کوانتومی، چندین شکاف در این حوزه وجود دارد که نیاز به توجه دارند. یکی از شکاف‌های برجسته، کمبود محک‌های جامع و عملکردگرا برای شبیه‌سازهای کوانتومی است. باوجود محک‌های متعدد 
\cite{michielsen_benchmarking_2017, wright_benchmarking_2019, koch_demonstrating_2020, mills_application-motivated_2021, cornelissen_scalable_2021}،
اغلب، گسترهٔ کاربردهای عملی پیش‌بینی‌شده برای رایانش کوانتومی را پوشش نمی‌دهند. کاربردهای پیش‌بینی‌شده شامل جست‌وجو در پایگاه‌داده‌های مرتب‌نشده، فاکتورکردن اعداد بزرگ به عوامل اول و مسائل بهینه‌سازی با تعداد متغیرهای قابل‌توجه هستند که اهمیت وجود رایانه‌های کوانتومی را در مسائل واقعی نشان می‌دهند. این شکاف نشان می‌دهد که نیاز به مجموعه‌ای متنوع‌تر از محک‌ها وجود دارد که منعکس‌کنندهٔ موارد استفاده واقعی باشند و بتوانند معیار دقیق‌تری از قابلیت‌های یک سامانه کوانتومی را ارائه دهند. 
محک‌های فعلی نمی‌توانند انواع شبیه‌سازها را در یک چارچوب بررسی کنند. چارچوب به این معنا که محک‌ها از یک اصول برای پیاده‌سازی و اعمال روی شبیه‌سازها پیروی کنند. دلیل این عدم توانایی این است که شبیه‌سازها از رویکردها و روش‌های متفاوت برای شبیه‌سازی استفاده می‌کنند. بعضی شبیه‌سازها فقط برای شبیه‌سازی یک الگوریتم طراحی شده‌اند در حالی که بعضی دیگر برای شبیه‌ساز هر الگوریتمی طراحی شده‌اند.
برای مثال شبیه‌ساز
\cite{noauthor_quantum_nodate-1}
به طور خاص برای شبیه‌سازی الگوریتم Shor طراحی شده است که محک آن با شبیه‌سازی‌هایی که توانایی شبیه‌سازی هر الگوریتمی را دارند متفاوت خواهد بود. از دیگر تفاوت‌هایی که می‌توان به آن اشاره کرد، محک شبیه‌سازها با استفاده از ابررایانه‌ها و یا مراکز محاسبات پردازش سریع است که نتایج بی‌فایده‌ای را برای کاربران قابل‌توجهی که شبیه‌سازها را بر روی رایانه‌های شخصی اجرا می‌کنند، تولید کرده‌اند 
\cite{jamadagni_benchmarking_2024}.
این نتایج فقط برای کاربرانی مفید است که به ابررایانه‌ها و مراکز پردازش سریع دسترسی دارند.

همان‌طور که اشاره شد، محک‌های فعلی نتوانستند یک چارچوب برای محک همهٔ انواع شبیه‌سازی را طراحی کنند. این ناسازگاری، توانایی مقایسهٔ عادلانهٔ سامانه‌های مختلف را مختل می‌کند. پرداختن به این مسئله، شامل توسعهٔ محک‌هایی است که بتوانند به طور مشابه از لحاظ پیاده‌سازی و الگوریتمی (به کارگیری گیت‌ها) در پلتفرم‌های مختلف اعمال شوند و در نتیجه، ارزیابی‌های دقیق‌تر و منصفانه‌تری از فناوری‌های رایانش کوانتومی را به ارمغان بیاورند
 \cite{lubinski_application-oriented_2023}.
در پژوهش ما سعی شده است که باتوجه‌ به شرایط و امکانات موجود تا حد توان به این هدف پرداخته شود؛ اما به دلیل پیچیدگی زیاد مسئله، هنوز مشکلاتی از آن برطرف نشده است. هدف طراحی محک‌هایی است که با کم‌ترین تغییرات بتوان آن را بر روی همه شبیه‌سازها اعمال کرد. به همین دلیل، نیاز است روشی برای تبدیل تمامی الگوریتم‌ها به یک ساختار واحد که قابل‌پذیرش برای همهٔ شبیه‌سازها باشد، ارائه شود. با وجود تمامی پیچیدگی‌های این مسئله، برای پیداکردن چنین روشی در پژوهش ما گام برداشته شده است؛ این روش به طراحی یک واحد مترجم برای ترجمهٔ الگوریتم‌ها از یک‌زبان مبدأ به زبان مقصد که همان شبیه‌سازهای مختلف باشند، می‌پردازد. این در حالی است که تعداد زیاد شبیه‌سازها و عدم پیروی آن‌ها از یک رویکرد یکسان، طراحی این مترجم را پیچیده و مشکل می‌سازد.

\section{روش پیشنهادی}
در پژوهش ما برای بهبود ضعف‌های موجود، مبناهای محک شبیه‌ساز به‌طورکلی متفاوت و جامع در نظر گرفته شده است. کارهای پیشین، اغلب به محک رایانه‌های واقعی کوانتومی پرداخته‌اند و توجه خاص به شبیه‌سازها کم‌تر دیده می‌شود. در اندک پژوهش‌هایی که به شبیه‌سازها نیز توجه کرده‌اند، معیارهای انتخاب‌شده برای محک، معیارهای مطلوبی در نظر گرفته نشده است. مطلوب از این بابت است که معیارها عموماً برای محک سخت‌افزار و عملکرد گیت‌ها به‌صورت مجزا انتخاب شده‌اند. معیارهای مطلوب معیارهایی هستند که برای محک شبیه‌سازها در زمان اجرای الگوریتم‌ها و بررسی عملکرد آن‌ها به‌طوری‌که همهٔ جنبه‌های مربوطه از تبدیل محاسبات کوانتومی به محاسبات کلاسیک تا ذخیره‌ٔ اطلاعات بدست آمده، در آن‌ها بررسی شود. علاوه‌بر معیار، با تعریف روش‌های بهتر محک، می‌توان به معیارهای مهم نظیر زمان اجرا و میزان مصرف حافظه، توجه ویژه‌ای داشت. در روش‌های محک شبیه‌سازها نیز کم‌تر به الگوریتم‌هایی که مسائل واقعی را حل می‌کنند توجه شده است درصورتی‌که مبنای محک در پژوهش ما انتخاب الگوریتم‌هایی بوده است که از آن در حل مسائل واقعی استفاده می‌شود. علاوه بر این‌ها، فرض پژوهش‌های پیشین بر آن بوده است که کاربران شبیه‌سازها تماماً به رایانه‌های پردازش سریع و یا ابررایانه‌ها دسترسی فراوان دارند و محک‌های خود را در چنین ساختارهایی انجام داده‌اند درصورتی‌که در پژوهش ما کاربرانی در نظر گرفته شده‌اند که شبیه‌سازی‌ها را در رایانهٔ شخصی معمولی انجام می‌دهند و به ابررایانه‌ها و یا رایانه‌های کوانتومی واقعی دسترسی ندارند. در کنار تمامی موارد اشاره شده، در طول پژوهش، ابزاری\LTRfootnote{\url{https://github.com/ahmadrv/QSB}} توسعه داده شده است که پژوهش‌های آینده در این زمینه را به سمت استانداردسازی هدایت می‌کند و فرآیند محک را آسان‌تر و سریع‌تر به پیش می‌برد.

\section{دستاوردها}
در پژوهش ما ابزار
\lr{Quantum Simulator Benchmark (QSB)}
برای محک شبیه‌سازهای کوانتومی، طراحی شده است. بعد از انتخاب محیط آزمایش، یعنی همان رایانهٔ مورداستفاده برای شبیه‌سازی، کافی است QSB نصب و اجرا شود تا باتوجه‌به رایانه‌ای که در حال اجرای آن است، نمودارها و جدول‌های مربوط به زمان اجرا و میزان حافظهٔ مصرفی، تولید شوند. این ابزار قابلیت این را دارد که الگوریتم‌های جدید پیاده‌سازی شده در هر شبیه‌ساز را در خود جای دهد و با معیارهای مربوط، توانایی شبیه‌ساز در اجرای آن‌ها را محک بزند. با استفاده از این ابزار، تعدادی آزمایش بر روی تعدادی از شبیه‌سازهای مشهور انجام شده که نشان‌دهندهٔ رفتار متفاوت شبیه‌سازها در زمان اجرای محک‌های مختلف است. برای مثال شبیه‌ساز
\lr{Cirq}،
 در اجرای الگوریتم
\lr{Grover}،
بسیار سریع‌تر از Simon عمل می‌کند. همچنین، به طور شگفت‌آوری، با بررسی این آزمایش‌ها و مقایسهٔ آن‌ها با خروجی‌های گزارش‌شده‌ای که بر روی رایانه‌های پردازش سریع اجرا شده است
\cite{jamadagni_benchmarking_2024}،
می‌توان متوجه شد که افزایش قدرت سخت‌افزاری بهبود نسبی چندانی در شبیه‌سازی الگوریتم‌ها با افزایش تعداد کیوبیت‌ها ندارد به این معنا که اگر چه قدرت محاسباتی چه در افزایش حجم حافظه و تعداد هسته‌های پردازشی چندین برابر شده اما مدت‌زمان اجرا به همان نسبت کاهش نیافته است. در فصل آخر، به طور کامل، به توصیف رفتارهای شبیه‌سازها در مواجه با اجرای الگوریتم‌های مختلف می‌پردازیم. در نتیجه، دستاوردهای  پژوهش ما به شرح زیر است:
\begin{enumerate}[label=\arabic*{-}]
	\item محک دقیق‌تر و تخصصی‌تر شبیه‌سازهای پراستفاده در مقایسه با دیگر محک‌ها
	\item تولید یک ابزار برای کارآمد کردن و آسان‌سازی فرایند محک بر روی رایانه‌های شخصی
\end{enumerate}

\section{نتایج ابتدایی}
همان‌طور که اشاره شد، با استفاده از ابزار توسعه داده شده، آزمایش‌های متفاوتی بر روی شبیه‌سازهای مختلف انجام شده است به‌طوری‌که برای مثال زمان اجرای الگوریتم‌های مختلف بر روی شبیه‌ساز Cirq باتوجه‌به افزایش تعداد کیوبیت‌ها، در شکل
\ref{fig:1}
قابل مقایسه است.
\begin{figure}
	\centering
	\captionsetup{justification=centering,margin*=0.5cm}
	% This file was created with tikzplotlib v0.10.1.
\begin{tikzpicture}

	\definecolor{chocolate2267451}{RGB}{226,74,51}
	\definecolor{dimgray85}{RGB}{85,85,85}
	\definecolor{gainsboro229}{RGB}{229,229,229}
	\definecolor{gray119}{RGB}{119,119,119}
	\definecolor{lightgray204}{RGB}{204,204,204}
	\definecolor{mediumpurple152142213}{RGB}{152,142,213}
	\definecolor{sandybrown25119394}{RGB}{251,193,94}
	\definecolor{steelblue52138189}{RGB}{52,138,189}
	\definecolor{yellowgreen14218666}{RGB}{142,186,66}

	\begin{axis}[
			axis line style={black},
			tick style={black},
			grid=major,
			legend cell align={left},
			legend style={
					fill opacity=0.6,
					draw opacity=1,
					text opacity=1,
					at={(0.03,0.97)},
					anchor=north west,
					draw=lightgray204,
					fill=gainsboro229,
					font=\tiny,
					scale=0.1,
				},
			log basis y={10},
			tick align=outside,
			tick pos=left,
			x grid style={white},
			xlabel={\scriptsize\textcolor{black}{\rl{کیوبیت (عدد)}}},
			xmajorgrids,
			xmin=1, xmax=30,
			xtick distance=3,
			width=15cm,
			height=8cm,
			xtick style={color=dimgray85},
			y grid style={gray119},
			ylabel={\scriptsize\textcolor{black}{\rl{زمان اجرا (ثانیه)}}},
			ymajorgrids,
			ymin=0, ymax=3601,
			ymode=log,
			ytick style={color=dimgray85},
			ytick={1,10,100,1000,3600},
			yticklabels={
					\(\displaystyle {10^{0}}\),
					\(\displaystyle {10^{1}}\),
					\(\displaystyle {10^{2}}\),
					\(\displaystyle {10^{3}}\),
					\(\displaystyle {3600}\),
				},
			tick label style={font=\scriptsize}
		]
		\addplot [semithick, chocolate2267451, dash pattern=on 1.5pt off 2.475pt, mark=*, mark size=3, mark options={solid}]
		table {%
				2 1.48365872234376
				3 1.51706652833166
				4 1.52457615372802
				5 1.51269077919883
				6 1.53364316867795
				7 1.60500001020549
				8 1.7119184330712
				9 1.86577925330689
				10 2.33818849931145
				11 3.13396421108825
				12 5.0417146568137
				13 9.43095060943771
				14 19.7935344980885
				15 45.3588669744664
				16 117.958052306411
				17 365.210204124495
				18 1499.49220539235
			};
		\addlegendentry{Deutsch-Jozsa (balanced)}
		\addplot [semithick, steelblue52138189, dash pattern=on 1.5pt off 2.475pt, mark=*, mark size=3, mark options={solid}]
		table {%
				2 1.52068267613627
				3 1.49706569139267
				4 1.54232655035433
				5 1.4944052702913
				6 1.52773775547098
				7 1.5128114824183
				8 1.53947932412141
				9 1.48252398054064
				10 1.52835301513491
				11 1.48240306277457
				12 1.52001491420359
				13 1.52712670238701
				14 1.50656203300293
				15 1.51695590659162
				16 1.458316882073
				17 1.52201066120314
				18 1.49419679988669
				19 1.4802567407604
				20 1.4899878464422
				21 1.5241209164974
				22 1.49416006494611
				23 1.4985360259794
				24 1.50484482585533
				25 1.50751394334611
				26 1.52429736514094
				27 1.51326132763924
				28 1.49795787272235
				29 1.51282007081748
			};
		\addlegendentry{Deutsch-Jozsa (constant)}
		\addplot [semithick, mediumpurple152142213, dash pattern=on 1.5pt off 2.475pt, mark=*, mark size=3, mark options={solid}]
		table {%
				2 1.46274263762187
				3 1.51281461088441
				4 1.48153519304691
				5 1.53522677650982
				6 1.50748089243174
				7 1.46746077772961
				8 1.5126636746496
				9 1.52057290353388
				10 1.52849783690534
				11 1.48473682466646
				12 1.54701619005683
				13 1.50252877136257
				14 1.51118434592731
				15 1.51475692696991
				16 1.48237675383404
				17 1.50453540415802
				18 1.51677126736144
				19 1.50191968297902
				20 1.51411929230592
				21 1.55205083615833
				22 1.4973890291953
				23 1.50806513820737
				24 1.50611302782774
				25 1.52539227105839
				26 1.475959476947
				27 1.51709598397667
				28 1.4903128921372
				29 1.57184540569696
			};
		\addlegendentry{Bernstein-Vazirani (basic)}
		\addplot [semithick, gray119, dash pattern=on 1.5pt off 2.475pt, mark=*, mark size=3, mark options={solid}]
		table {%
				2 1.53186994940765
				3 1.48521215930964
				4 1.48373876934692
				5 1.49738585662536
				6 1.49549151139647
				7 1.5096549595808
				8 1.5390604825866
				9 1.48954580897664
				10 1.51642010482938
				11 1.47222273684598
				12 1.52271072510916
				13 1.56584157135331
				14 1.59020704391039
				15 1.60959737718139
				16 1.62607432416971
				17 1.60181091714048
				18 1.69592118479684
				19 1.77558230830292
				20 2.03857360881322
				21 2.60831062112275
				22 3.81965398122993
				23 6.32561060039173
				24 11.3944282313533
				25 21.8525129944648
				26 43.5642095983073
				27 89.0492528337198
			};
		\addlegendentry{Quantum Fourier Transform (basic)}
		\addplot [semithick, sandybrown25119394, dash pattern=on 1.5pt off 2.475pt, mark=*, mark size=3, mark options={solid}]
		table {%
				2 5.47442678527333
				3 5.65824037110087
				4 6.37251109332659
				5 7.0574273855132
				6 7.74969564306367
				7 8.4327082937681
				8 9.10634736100935
				9 10.1888861090543
				10 10.7773838986736
				11 11.3576426655198
				12 12.7662866945969
				13 13.3889331781312
				14 14.2005654201978
				15 20.0137540101241
				16 21.1128365545457
				17 144.095285251771
				18 31.125176478255
				19 161.288181415424
				20 70.6107039671483
				21 542.893891105844
				22 65.1499652167538
				23 75.6698851216986
				24 143.742745176782
				25 490.430703527016
				26 1266.80518196843
				27 47.815201878588
				28 110.23032660909
				29 1137.73970382624
			};
		\addlegendentry{Simon (basic)}
		\addplot [semithick, yellowgreen14218666, dash pattern=on 1.5pt off 2.475pt, mark=*, mark size=3, mark options={solid}]
		table {%
				2 1.51330135402936
				3 1.52983067825293
				4 1.48492169159033
				5 1.55987351702811
				6 1.57019396571298
				7 1.59605143231454
				8 1.56125774241925
				9 1.55435996228315
				10 1.57862519714227
				11 1.61289317569266
				12 1.58426994151543
				13 1.63825136870776
				14 1.63821306417599
				15 1.57294436466368
				16 1.57367301818906
				17 1.60577940761527
				18 1.63009363301096
				19 1.59271337665299
				20 1.59898629893113
				21 1.58910673406072
				22 1.60741208556624
				23 1.61081145855502
				24 1.62224761932895
				25 1.59337108418009
				26 1.6266805163698
				27 1.62334704916834
				28 1.63813540961341
				29 1.60653722719372
			};
		\addlegendentry{Grover (basic)}
	\end{axis}
\end{tikzpicture}

	\caption[نمودار زمان اجرا Cirq]{
	نمودار زمان اجرا بر حسب تعداد کیوبیت در الگوریتم‌های مختلف در شبیه‌ساز Cirq}
	\label{fig:1}
\end{figure}

در این نتایج، رفتارهای متفاوتی از شبیه‌سازها مختلف دیده می‌شود به‌گونه‌ای که می‌توان آن‌ها را نسبت به ادعایی که در تئوری داشته‌اند مقایسه کرد. این مقایسه در فصل آخر مورد بررسی قرار گرفته شده است.