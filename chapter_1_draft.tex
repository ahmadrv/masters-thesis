\section{چگونگی تولد رایانش کوانتومی و اهداف آن}
ظهور رایانش کوانتومی به اوایل دهه ۱۹۸۰ بازمی‌گردد، زمانی که فیزیک‌دان، ریچارد فاینمن\LTRfootnote{Richard Feynman}، ایده‌ی استفاده از مکانیک کوانتومی برای انجام محاسبات را مطرح کرد. فاینمن مشاهده کرد که رایانه‌های کلاسیک در شبیه‌سازی سیستم‌های کوانتومی با چالش‌های قابل توجهی مواجه هستند، زیرا پیچیدگی مرتبط با حالت‌های کوانتومی به صورت نمایی افزایش می‌یابد. این بینش منجر به پایه‌گذاری مفهومی تحت عنوان «رایانش کوانتومی» شد که هدف آن استفاده از پدیده‌های مکانیکی کوانتومی برای پردازش اطلاعات با کارایی بیش‌تری نسبت به رایانه‌های کلاسیک است. رایانش کوانتومی، بهره‌برداری از اصول برهم‌نهی، درهم‌تنیدگی و تداخل کوانتومی برای حل مسائل پیچیده‌ای است که برای رایانه‌های کلاسیک غیرقابل حل هستند. رایانه‌های کوانتومی برای انجام وظایفی مانند فاکتورگیری اعداد بزرگ، جستجو در پایگاه‌های مه‌داده‌ها و شبیه‌سازی فرآیندهای فیزیکی کوانتومی طراحی شده‌اند که با سرعت بسیار بیش‌تری نسبت به همتایان کلاسیک خود این فعالیت‌ها را انجام می‌دهند. این پتانسیل برای افزایش توان محاسباتی، باعث تحقیقات و توسعه مداوم در این زمینه شده است با این امید که پیشرفت‌های چشم‌گیری در حوزه‌های علمی و فناوری مختلف به دست آید \cite{nielsen_quantum_2010}.
\section{نقش شبیه‌سازهای مدار کوانتومی در پیش‌برد رایانش کوانتومی}
شبیه‌سازهای مدار کوانتومی نقش مهمی در پیش‌برد رایانش کوانتومی دارند زیرا بستری را برای طراحی، آزمایش و بهینه‌سازی الگوریتم‌های کوانتومی قبل از پیاده‌سازی آن‌ها روی سخت‌افزار واقعی کوانتومی فراهم می‌کنند. این شبیه‌سازها رفتار مدارهای کوانتومی را مدل‌سازی می‌کنند و به پژوهش‌گران این امکان را می‌دهند تا ویژگی‌ها و عملکرد الگوریتم‌های کوانتومی را تحت شرایط کنترل‌شده بررسی کنند. این قابلیت برای درک مزایای محاسباتی مکانیک کوانتومی و شناسایی مشکلات احتمالی در طراحی الگوریتم بسیار ضروری است \cite{nielsen_quantum_2010}.

یکی از وظایف شبیه‌سازهای مدار کوانتومی این است که شبیه‌سازی سیستم‌های کوانتومی را که برای رایانه‌های کلاسیک بسیار پیچیده هستند، امکان‌پذیر کنند. با استفاده از شبیه‌سازهای مدار کوانتومی، پژوهش‌گران می‌توانند درک بهتری از دینامیک الگوریتم‌های کوانتومی پیدا کنند، استراتژی‌های محاسباتی جدیدی را کشف کنند و رویکردهای خود را برای بهره‌برداری کامل از پتانسیل رایانش کوانتومی بهینه‌سازی کنند. این فرآیند تکراری شبیه‌سازی و بهینه‌سازی برای توسعه‌ی عملی فناوری‌های رایانش کوانتومی ضروری است و راه را برای سیستم‌های کوانتومی مقاوم‌تر و مقیاس‌پذیرتر در آینده، هموار می‌کند \cite{nielsen_quantum_2010}.
\section{پرسش‌های پژوهش}
\subsection*{معیارهای عملکرد کلیدی برای محک منصفانه‌ی شبیه‌سازهای مدار کوانتومی چیست؟}
\subsubsection{زمان شبیه‌سازی}
این زمان، زمان کلی مورد نیاز برای شبیه‌سازی یک مدار کوانتومی است. این معیار بسیار مهم است زیرا بازتابی از کارایی و سرعت شبیه‌ساز می‌باشد. عواملی که بر زمان شبیه‌سازی تأثیر می‌گذارند شامل پیچیدگی مدار کوانتومی، تعداد کیوبیت‌ها و ماهیت الگوریتمی است که شبیه‌سازی می‌شود\cite{lubinski_application-oriented_2023, jamadagni_benchmarking_2024}.

\subsubsection{مصرف حافظه}
اندازه‌گیری مقدار حافظه‌ای که برای انجام شبیه‌سازی مورد نیاز است از دیگر معیارهای مهم به شمار می‌رود. این معیار به ویژه برای شبیه‌سازی‌های بزرگ‌مقیاس که محدودیت‌های حافظه می‌تواند به یک عامل محدودکننده تبدیل شود، اهمیت دارد. به دلیل رشد نمایی فضای حالت کوانتومی با افزایش تعداد کیوبیت‌ها، شبیه‌سازهای کوانتومی، اغلب نیاز به مدیریت مقادیر زیادی از داده‌ها دارند \cite{lubinski_application-oriented_2023}.

\subsubsection{قابلیت مقیاس‌پذیری}
توانایی شبیه‌ساز برای مدیریت افزایش تعداد کیوبیت‌\LTRfootnote{qubit}ها و عمق مدار\LTRfootnote{circuit depth}، قابلیت مقیاس‌پذیری نامیده می‌شود. عمق مدار، همان تعداد لایه‌هایی است که میزان پیچیدگی مدارهای کوانتومی را می‌سنجد. مقیاس‌پذیری برای کاربردهای عملی شبیه‌سازی کوانتومی، حیاتی است زیرا تعیین می‌کند که شبیه‌ساز چگونه می‌تواند مشکلات واقعی با اندازه و پیچیدگی قابل توجه را مدیریت کند \cite{jamadagni_benchmarking_2024, lubinski_application-oriented_2023}.
\subsubsection{دقت}
میزان دقتی که شبیه‌ساز می‌تواند رفتار یک سیستم کوانتومی را تکرار کند نیز باید مورد محک قرار بگیرد. این، شامل چگونگی برخورد شبیه‌ساز با نویز کوانتومی، کاهش ناهمدوسی\LTRfootnote{decoherence} و سایر اثرات کوانتومی است که بر صحت نتایج شبیه‌سازی تأثیر بسزایی می‌گذارند \cite{jamadagni_benchmarking_2024, lubinski_application-oriented_2023}.
\subsubsection{انعطاف‌پذیری}
از توانایی شبیه‌ساز برای پشتیبانی از الگوریتم‌های مختلف کوانتومی و انواع مختلف مدارهای کوانتومی به انعطاف‌پذیری یاد می‌شود. انعطاف‌پذیری شامل پشتیبانی از زبان‌های برنامه‌نویسی مختلف و ادغام با چارچوب‌های مختلف رایانش کوانتومی است \cite{jamadagni_benchmarking_2024, lubinski_application-oriented_2023, young_simulating_2023}.
\subsubsection{پیچیدگی پیاده‌سازی الگوریتم‌های کوانتومی}
پیاده‌سازی الگوریتم‌ها در شبیه‌سازهای مختلف، متفاوت است، به گونه‌ای که هر کدام با توجه به تکنولوژی مورد استفاده مثل زبان برنامه‌نویسی، قابلیت اجرا بر روی سخت‌افزارهای خاص، دارای پیچیدگی‌های منحصربه‌فرد هستند. این معیار، ترکیبی از زمان یادگیری نحوه‌ی استفاده از شبیه‌ساز و زمان پیاده‌سازی یک الگوریتم در آن می‌باشد که با توجه به این که به توانایی افراد وابسته است، یک معیار کیفی است اما به دلیل تأثیر زیاد آن در روند پیاده‌سازی، لایق بررسی است.

\subsection*{عملکرد شبیه‌سازهای مختلف بر روی الگوریتم‌های مختلف کوانتومی چگونه است؟}
شبیه‌سازهای مختلف مدار کوانتومی، عملکرد متفاوتی بر روی الگوریتم‌های کوانتومی مختلف دارند. قسمت‌های زیر ویژگی‌های عملکرد‌های مشاهده شده را خلاصه می‌کنند.
\subsubsection{شبیه‌سازهای مبتنی بر شرودینگر}
این شبیه‌سازها که بر پایه‌ی نمایش مستقیم بردار حالت، استوار هستند. به دلیل پیاده‌سازی ساده و کاربرد عمومی، به طور گسترده‌ای استفاده می‌شوند. با این حال، آن‌ها اغلب در مقیاس‌بندی به بیش از یک تعداد خاص از کیوبیت‌ها به دلیل نیازهای حافظه‌ای نمایی با چالش‌هایی مواجه می‌شوند \cite{young_simulating_2023}.
\subsubsection{شبیه‌سازهای مبتنی بر شبکه‌ی تنسوری}
این شبیه‌سازها در مدارهای کوانتومی با محدودیت درهم‌تنیدگی، خوب عمل می‌کنند. با نمایش حالت‌های کوانتومی به عنوان شبکه‌های تنسوری، می‌توانند برخی از دسته‌های مدارهای کوانتومی را که با روش‌های مبتنی بر شرودینگر غیرقابل حل هستند، به طور کارآمد شبیه‌سازی کنند. با این حال، ممکن است با مدارهایی که شامل درجات بالایی از درهم‌تنیدگی هستند، دچار مشکل شوند \cite{young_simulating_2023}.
\subsubsection{روش‌های ترکیبی}
این روش‌ها، ترکیبی از تکنیک‌های شبیه‌سازی مختلف را برای بهره‌برداری از نقاط قوت هر یک ترکیب می‌کنند. به عنوان مثال، ادغام شبکه‌های تنسوری با روش‌های مبتنی بر شرودینگر می‌تواند تعادلی بین استفاده از حافظه و کارایی زمانی ایجاد کند که آن‌ها را برای طیف وسیع‌تری از مدارهای کوانتومی مناسب می‌سازد \cite{young_simulating_2023}.
\subsubsection{شبیه‌سازها با کاربرد خاص}
شبیه‌سازهایی که برای انواع خاصی از مدارها یا الگوریتم‌های کوانتومی بهینه‌سازی شده‌اند، مانند مدارهای پایدارکننده\LTRfootnote{Stabilizer Circuits} یا الگوریتم‌های کوانتومی متغیر\LTRfootnote{Variational Quantum Algorithms (VQA)}، در دامنه‌های مربوط به خود، عملکرد بهتری نسبت به شبیه‌سازهای عمومی دارند. این شبیه‌سازهای خاص می‌توانند به طور قابل توجهی زمان شبیه‌سازی و نیازهای منابع را برای کاربردهای هدف خود، کاهش دهند \cite{young_simulating_2023}.
\subsubsection{شبیه‌سازها با سخت‌افزار خاص}
شبیه‌سازهایی که برای بهره‌برداری از ویژگی‌های خاص سخت‌افزاری، مانند پردازنده‌های گرافیکی، طراحی شده‌اند، می‌توانند به بهبودهای قابل توجهی در عملکرد دست یابند. این شبیه‌سازها از قابلیت‌های پردازش موازی برای مدیریت سیستم‌های کوانتومی بزرگ‌تر با کارایی بیشتر نسبت به شبیه‌سازهای مبتنی بر CPU سنتی استفاده می‌کنند \cite{young_simulating_2023, xu_herculean_2023}.


\subsection*{نقاط قوت و ضعف شبیه‌سازهای فعلی در چه قسمت‌هایی است؟}
\subsubsection{نقاط قوت}
\paragraph{دقت بالا:}
بسیاری از شبیه‌سازهای فعلی، شبیه‌سازی‌هایی با دقت بالا ارائه می‌دهند که برای آزمایش و اعتبارسنجی الگوریتم‌های کوانتومی ضروری است. آن‌ها می‌توانند رفتار سیستم‌های کوانتومی را به‌طور دقیقی تکرار کنند که برای تحقیقات و توسعه در رایانش کوانتومی بسیار مهم است \cite{young_simulating_2023}.
\paragraph{عمومیت:}
شبیه‌سازهای مدرن از طیف وسیعی از الگوریتم‌ها و کاربردهای کوانتومی پشتیبانی می‌کنند، از دروازه‌های کوانتومی پایه تا کدهای پیچیده تصحیح خطای کوانتومی. این عمومیت، آن‌ها را به ابزارهای ارزشمندی برای حوزه‌های مختلف تحقیقات رایانش کوانتومی تبدیل کرده است \cite{xu_herculean_2023}.
\paragraph{ویژگی‌های پیشرفته:}
برخی از شبیه‌سازها ویژگی‌های پیشرفته‌ای مانند مدل‌سازی نویز، شبیه‌سازی تصحیح خطا و پشتیبانی از الگوریتم‌های کوانتومی متغیر را ارائه می‌دهند که برای شبیه‌سازی‌های واقع‌گرایانه و توسعه رایانش کوانتومی مقاوم به خطا، ضروری هستند \cite{jamadagni_benchmarking_2024}.
\subsubsection{نقاط ضعف}
\paragraph{مشکلات مقیاس‌پذیری:}
بسیاری از شبیه‌سازها با مقیاس‌بندی به تعداد بیش‌تری از کیوبیت‌ها به دلیل افزایش نمایی در نیازهای حافظه و محاسبات، با چالش‌هایی مواجه هستند. این محدودیت‌ها استفاده آن‌ها را برای سیستم‌های کوانتومی بزرگ‌تر و مسائل واقعی محدود می‌کند \cite{xu_herculean_2023}.
\paragraph{مصرف منابع زیاد:}
استفاده بالا از حافظه و زمان طولانی شبیه‌سازی چالش‌های رایجی هستند. شبیه‌سازی مدارهای بزرگ کوانتومی اغلب به منابع محاسباتی قابل توجهی نیاز دارد که برای بسیاری از کاربران بدون دسترسی به امکانات محاسباتی با کارایی بالا، عملی نیست \cite{young_simulating_2023}.
\paragraph{محدودیت‌های تخصصی:}
در حالی که شبیه‌سازهای خاص در حوزه‌های خود عالی هستند، اغلب فاقد انعطاف‌پذیری برای مدیریت انواع مختلف مدارهای کوانتومی هستند. این نیاز به استفاده از چندین شبیه‌ساز برای پوشش جنبه‌های مختلف رایانش کوانتومی را ایجاد می‌کند که می‌تواند فرآیند شبیه‌سازی را بسیار زمان‌بر و پیچیده کند \cite{jamadagni_benchmarking_2024}.
\paragraph{پیچیدگی:}
پیچیدگی راه‌اندازی و اجرای شبیه‌سازی‌ها، می‌تواند مانعی برای ورود کاربران جدید باشد. دانش دقیقی از اصول رایانش کوانتومی و ساختار خاص شبیه‌ساز اغلب برای دستیابی به عملکرد بهینه، مورد نیاز است \cite{jamadagni_benchmarking_2024}.
